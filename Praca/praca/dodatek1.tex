\chapter{Zawartość płyty CD}
\thispagestyle{chapterBeginStyle}
\label{plytaCD}

Na płycie CD znajduje się kod źródłowy programu.
Wybudowana gra nie zmieściła się w zakresie 100MB załącznika w systemie ASAP.\\% oraz Windows.\\
Aby otworzyć kod źródłowy, naleźy:
\begin{itemize}
    \item zainstalować silnik Unity zgodnie z instrukcjami z oficjalnej strony\\($https://unity.com/download$)
    \item zainstalować pakiet Ml-Agents\\($https://github.com/Unity-Technologies/ml-agents/blob/main/docs/Installation.md$)
    \item uruchomic Unity
    \item podmienić folder $Assets$ na folder dołączony w dodatku
    \item zaimportować poniższe pakiery z $UnityAssetStore$
        \subitem Best Sports CARS - Pro 3D Models
        \subitem Coconut Palm Tree Pack
        \subitem Conifers [BOTD]
        \subitem Standard Assets
        \subitem TextMesh Pro
    \item wybrać projekt \textit{game-code}
    \item załadować scenę $Scenes/MainMenu$, oraz uruchomić
\end{itemize}

Dodatkowo pełny kod źródłowy dostępny jest pod linkiem:\\
https://github.com/KamilMatejuk/Praca-Inzynierska-Gra

% \phantom{.}\\
% Aby uruchomić wybudowaną grę w systemie Windows należy uruchomić plik $game/game\_win/Racing Game.exe$.
% \phantom{.}\\
% Aby uruchomić wybudowaną grę w systemie Linux należy uruchomić plik $game/game\_linux/game.x86\_64$.
