\chapter*{Podsumowanie}
\addcontentsline{toc}{chapter}{Podsumowanie}
\thispagestyle{chapterBeginStyle}
W powyższej pracy została stworzona gra wyścigowa, wraz z botem. Bot został wytrenowany z wykorzystaniem uczenia przez wzmacnianie przy pomocy Unity ML-Agents. Jako interakcję ze środowiskiem bot posiada do dystpozycji dwie akcje w przestrzeni dyskretnej o wartościach {-1, 0, 1} definiujące poruszanie się do przodu/tyłu oraz kąt skrętu kierownicy. Pojazd porusza się zgodnie z uproszczonymi zasadami fizyki symulowanymi przez silnik Unity. Jako obserwacje przyjmuje odległości pojazdu od ewentualnych przeszkód oraz krawędzi drogi i aktualną prędkość. Taki zbiór danych wejściowych ma na celu symulować podejście wykorzystywane aktualnie w samochodach autonomicznych.

Proces implementacji bota, wraz z wyborem obserwacji, podejmowanych akcji, funkcji nagrody i wyboru hiperparametrów przedstawiony został na wykresach ilustrujących efekty podjętej decyzji. W pracy zostały również opisane kolejne etapy treningu bota.
