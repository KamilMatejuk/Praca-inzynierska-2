\begin{streszczenie}
    W poniższej pracy został opisany proces implementacji bota poruszającego się w środowisku gry wyścigowej. Gra pozwala na proceduralne generowanie tras i terenów, co pozwoliło stworzyć nieskończoną ilość środowisk testowych. Do treningu bota wykorzystano metodę uczenia maszynowego przez wzmacnianie. W opracowaniu przeanalizowane zostały różne rodzaje obserwacji, akcji oraz nagród. Finalnie bot został wytrenowany na podstawie danych o odległości pojazdu od ewentualnych przeszkód, symulując metodę używaną na co dzień w pojazdach autonomicznych - tzw. LIDAR \cite{Lidar}.
\end{streszczenie}

\vspace*{1cm}

\begin{abstract}
    The following thesis describes a process of implementing a racing bot, which has been taught to navigate a racing track environment. The game allows the user to procedurally create a track and a terrain, as well as play created levels. The bot has been trained using the reinforcement learning. This thesis analyzes different observations, action types and reward functions, to finally use distance-based observations. This method is similar to how autonomous vehicles operate in the real world - using LIDAR technology \cite{Lidar}.
\end{abstract}