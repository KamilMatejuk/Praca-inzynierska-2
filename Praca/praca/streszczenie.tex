\begin{streszczenie}
    W poniższej pracy został opisany proces implementacji bota poruszającego się w środowisku gry wyścigowej. Gra pozwala na proceduralne generowanie tras i terenów, co pozwoliło stworzyć nieskończoną ilość środowisk testowych. Do treningu bota wykorzystano metodę uczenia maszynowego przez wzmacnianie. W opracowaniu przeanalizowane zostały różne rodzaje obserwacji, akcji oraz nagród. Finalnie bot został wytrenowany na podstawie danych o odległości pojazdu od ewentualnych przeszkód, symulując metodę używaną na co dzień w pojazdach autonomicznych - tzw. lidar \cite{Lidar}.
\end{streszczenie}

\vspace*{1cm}

\begin{abstract}
    The following thesis describes a process of implementing racing bot, which has been taught to navigate racing track environment. Game allows user to procedurally create track and terrain, as well as play created levels. The bot has been trained using reinforcement learning. This thesis analyzes different observations, action types and reward functions, to finally use distance-based observations. This method is similar to how autonomous vehicles operate in real world - using lidar technology \cite{Lidar}.
\end{abstract}